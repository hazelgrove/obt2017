As terms change through evaluation, we rely on a notion of holes being
associated with substitutions, closures and unique names borrowed directly
from work in contextual modal type theory
(CMTT)~\cite{DBLP:journals/tocl/NanevskiPP08}. This allows a programmer to
view the trace as above and identify where each hole is used. CMTT is the
Curry-Howard interpretation of contextual modal logic, which gives us
confidence that our approach is rooted in logical tradition. CMTT does not
provide a way to reason about free metavariables, nor does it include the
notions of non-empty holes or holes in types which appear prominently in
Hazelnut, so there is work to be done to extend it to our use case.

Since complete terms coincide exactly with the simply typed lambda
calculus, it is important that any dynamic semantics for possibly
incomplete programs agrees with the standard treatment.

Progress states that a well-typed incomplete program will either step, is a
value, or is indeterminate. Preservation states that stepping preserves the
type of an incomplete program and that no new hole names are created by
evaluation.
\begin{conjecture}[Progress]
  If $\emptyset \vdash \hexp{} : \htau{} \dashv \Delta$ then either
  \begin{enumerate}[label=\roman*)]
  \item there exists $\hexp'$ such that $\pstep{\hexp{}}{\hexp{}'}$, or
  \item $\pval{\hexp}$, or
  \item $\pindet{\hexp}$
  \end{enumerate}
\end{conjecture}
\begin{conjecture}[Preservation]
  If $\emptyset \vdash \hexp{} : \htau{} \dashv \Delta$ and
  $\pstep{\hexp}{\hexp'}$ then $\emptyset \vdash \hexp'
  : \htau{} \dashv \Delta'$ and $\Delta' \subseteq \Delta$.
\end{conjecture}

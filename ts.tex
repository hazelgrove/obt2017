In order to state candidates for progress and preservation, we introduce a
new judgement on terms called \textit{final}. This is because there are
incomplete programs that we can't evalaute further that are not yet values,
i.e. $\hadd{9}{\hehole}$.

As terms evolve through their evaluation, we rely on a notion of holes
being associated with unique names and substitutions, borrowed directly
from work in contextual modal type theory
(CMTT). \cite{DBLP:journals/tocl/NanevskiPP08} Hole names correspond to
metavariables, which appear in $\Delta$ below, and holes themselves to
closures. CMTT is, in turn, the Curry-Howard interpretation of contextual
modal logic. This gives us confidence that our approach is not ad hoc, but
rather rooted in the established logical tradition.

Although CMTT provides a basis for our work, it does not provide a way to
reason about free metavariables, nor does it describe the notions of
non-empty holes or holes in types which appear prominently in Hazelnut.

\begin{conjecture}[Progress]
  If $\emptyset \vdash \hexp{} : \htau{} \dashv \Delta$ then either
  \begin{enumerate}[label=\roman*)]
  \item there exists $\hexp'$ such that $\pstep{\hexp{}}{\hexp{}'}$, or
  \item $\hexp ~\mathbf{final}$
  \end{enumerate}
\end{conjecture}

\begin{conjecture}[Preservation]
  If $\emptyset \vdash \hexp{} : \htau{} \dashv \Delta$ and
  $\pstep{\hexp}{\hexp'}$ then $\emptyset \vdash \hexp' : \htau{} \dashv
  \Delta'$ and $\Delta' \subseteq \Delta$.
\end{conjecture}

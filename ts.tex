%!TEX root = dynamics-obt17.tex
As terms change through evaluation, we rely on a notion of holes being
associated with substitutions, closures and unique names borrowed directly
from work on contextual modal type theory
(CMTT)~\cite{DBLP:journals/tocl/NanevskiPP08}. This allows a programmer to
view the trace as above and identify where each hole is used. CMTT is the
Curry-Howard interpretation of contextual modal logic, which gives us
confidence that our approach is rooted in the logical tradition. CMTT does
not provide a way to reason about the dynamics of terms with free
metavariables, nor does it include the notions of non-empty holes or holes
in types which appear prominently in Hazelnut, so there is work to be done
to extend it to our use case.

Progress states that a well-typed incomplete program will either step, is a
value, or is indeterminate.
\begin{conjecture}[Progress]
  If $\emptyset \vdash \hexp{} : \htau{} \dashv \Delta$ then either
  \begin{enumerate}[label=\roman*)]
  \item there exists $\hexp'$ such that $\pstep{\hexp{}}{\hexp{}'}$, or
  \item $\pval{\hexp}$, or
  \item $\pindet{\hexp}$
  \end{enumerate}
\end{conjecture}

Preservation states that stepping preserves the type of an incomplete
program and that no new hole names are created by evaluation.
\begin{conjecture}[Preservation]
  If $\emptyset \vdash \hexp{} : \htau{} \dashv \Delta$ and
  $\pstep{\hexp}{\hexp'}$ then $\emptyset \vdash \hexp'
  : \htau{} \dashv \Delta'$ and $\Delta' \subseteq \Delta$.
\end{conjecture}
Here, we need to explicitly consider the possibility of duplication of
holes, but ensure that they continue to appear at the same type and with
compatible closures.

Since complete terms coincide exactly with the simply typed lambda
calculus, it is important that any dynamic semantics and associated
metatheory for possibly incomplete terms agree with the standard
treatment. For example, complete programs should not become indeterminate
during evaluation, and standard type saftey should be provable for complete
programs without changing the small-step relation.

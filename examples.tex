%!TEX root = dynamics-obt17.tex

In our Hazelnut paper, incomplete expressions and types are governed by the
following grammar:
\begin{align*}
  \htau & ::=
  \tarr{\htau}{\htau} ~\vert~
  \tnum ~\vert~
  \tehole{u}
  \\
  \hexp & ::=   x ~\vert~
  \hlam{x}{\hexp} ~\vert~
  \hap{\hexp}{\hexp} ~\vert~
  \hnum{n} ~\vert~
  \hadd{\hexp}{\hexp} ~\vert~
  (\hexp : \htau) ~\vert~
  \hehole{u} ~\vert~
  \hhole{\hexp}{u}
\end{align*}
where $\hehole{u}$ and $\hhole{\hexp}{u}$ are holes. Each hole has a unique
name, $u$.

We write $\mapsto$ for the small-step relation. The judgement
$\pval{\hexp}$ is derivable when a $\hexp$ is a value, in the standard
sense. For example:
\begin{align*}
  \pval{(\hlam{x}{ \hadd{9}{\hehole{u}}})}
\end{align*}
The judgement $\pindet{\hexp}$ is derivable when $\hexp$ is
\textit{indeterminate}---when its evaluation cannot proceed because of a
hole in a critical place (roughly, in elimination position.) For example,
the following cannot continue to step, nor is it a value:
\begin{align*}
  \pindet{\hadd{9}{\hehole{u}}}
\end{align*}

Evaluation continues until the term in question is either a value or
indeterminate. Collectively, we call these \emph{final} evaluation states.

A function applied to a hole steps as usual:
\begin{align*}
      \pstep
      {
        \hap{\hlam{x}{\hadd{x}{\hehole{u}}}}{\hehole{u'}}
      }
      {
        \hadd{\hehole{u'}}{\hehole{u}}
      }
\end{align*}

Substitutions occur inside non-empty holes, even if the type of the
contents is not yet consistent with its surroundings. Otherwise, free
variables would appear in the result, breaking any familiar notion of
preservation. For instance, if $f : \tarr{\tnum}{\tarr{\tnum}{\tnum}}$,
\begin{align*}
  \pstep
      {
        \hap{\hlam{x}{\hadd{x}{\hhole{f~x}}{u}}}{5}
      }
      {
        \hadd{5}{\hhole{f~5}{u}}
      }
\end{align*}

\subsection{Extended Hazel}
To consider a more realistic example, we consider a hypothetical extension
of Hazelnut with features such as polymorphism, pattern matching, ADTs, and
recursive functions in ML-like syntax.

Imagine a programmer implementing $$\mathbf{map} : (\alpha \to \beta) \to
\alpha ~\mathit{list} \to \beta ~\mathit{list}$$ from the informal
specification that the function be applied to every list element uniformly.
She might arrive at the incomplete term
\begin{lstlisting}
fun map f [] = []
  | map f (x::xs) = $\hehole{u}$::(map f l)
\end{lstlisting}
if she knows the pattern of recursion, but is not sure what to do with each
element. Evaluating this incomplete program on a small example produces a
trace:
\begin{lstlisting}
   map ($\lambda x.$x+1) [1, 2]
$\mapsto^*$ $\hehole{u}$::(map ($\lambda x.$x+1) [2])
$\mapsto^*$ $\hehole{u}$::$\hehole{u}$::(map ($\lambda x.$x+1) [])
$\mapsto^*$ $\hehole{u}$::$\hehole{u}$::[]
\end{lstlisting}
Notice that a hole can become duplicated during evaluation. Each hole in
the trace has an associated environment with the overall goal type for the
hole and a mapping from in-scope variables names and their types to
run-time values. The environments for the two occurrences of the hole above
are in Table \ref{tabx}.

\begin{table}[h!]
  \begin{subtable}{.5\linewidth}
    \caption{hole type: $\beta=\texttt{int}$}
    \centering
        {
          \footnotesize
          \begin{tabular}{c|c|c}
            var & $\htau$ & val\\
            \hline
            \texttt{f} & $int \to int$ & $\lambda x.x$\texttt{+1}\\
            \texttt{l} & $int ~list$ & \texttt{[1,2]}\\
            \texttt{x} & $int$ & \texttt{1}\\
            \texttt{xs} & $int ~list$ & \texttt{[2]}\\
          \end{tabular}
        }
  \end{subtable}%
  \begin{subtable}{.5\linewidth}
    \caption{hole type: $\beta=\texttt{int}$}
    \centering
        {
          \footnotesize
          \begin{tabular}{c|c|c}
            var & $\htau$ & val\\
            \hline
            \texttt{f} & $int \to int$ & $\lambda x.x$\texttt{+1}\\
            \texttt{l} & $int ~list$ & \texttt{[2]}\\
            \texttt{x} & $int$ & \texttt{2}\\
            \texttt{xs} & $int ~list$ & \texttt{[]}\\
          \end{tabular}
        }
  \end{subtable}
  \caption{environment around each occurrence of $\hehole{u}$}
  \label{tabx}
\end{table}

This trace might lead the programmer to have more confidence about their
solution so far. An editor might display the associated environments during
the edit process, making it easier for the programmer to explore how to
fill in the hole. Here, following the types, the programmer might fill the
hole with $\hhole{\texttt{f x}}{u}$. If she makes this change and then
reruns her example, she will see that she has indeed satisfied the
specification, and should now replace the hole with its contents:
\begin{lstlisting}
   map ($\lambda x.$x+1) [1, 2]
$\mapsto^*$ $\hhole{\hap{(\lambda x.x+1)}{1}}{u}$::(map ($\lambda x.$x+1) [2])
$\mapsto^*$ $\hhole{\hadd{1}{1}}{u}$::(map ($\lambda x.$x+1) [2])
$\mapsto^*$ $\hhole{2}{u}$::(map ($\lambda x.$x+1) [2])
$\mapsto^*$ $\hhole{2}{u}$::$\hhole{\hap{\lambda x.x+1}{2}}{u}$::(map ($\lambda x.$x+1) [])
$\mapsto^*$ $\hhole{2}{u}$::$\hhole{\hadd{2}{1}}{u}$::(map ($\lambda x.$x+1) [])
$\mapsto^*$ $\hhole{2}{u}$::$\hhole{3}{u}$::(map ($\lambda x.$x+1) [])
$\mapsto^*$ $\hhole{2}{u}$::$\hhole{3}{u}$::[]
\end{lstlisting}

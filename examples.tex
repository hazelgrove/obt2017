To motivate the purpose of a dynamics for partial programs, we first
consider a series of examples. Some of these examples are within the language
of Hazelnut, but the later ones use ML-like syntax that isn't part of the
definition of the language yet. We intend to extend the language to include
such constructs, and believe an understanding of the dynamics of such terms
will help guide that extension.

These examples are written assuming an eager evaluation semantics. It's not
clear to us if that is important, or what the tradeoffs would be, so this
should be considered a stylistic detail that is not set in stone.

Since complete terms coincide exactly with the simply typed lambda
calculus, the evaluation semantics for them are totally
standard. Therefore, all the examples here are of terms with holes in
them. We explore the space of what could happen to partial terms during
evaluation by considering different scenarios of where the holes appear and
how they're filled.

\subsection{Current Hazelnut}
Like in standard lambda calculus, function literals do not
evaluate further, reguardless of their contents:
\begin{align*}
  \pval{\hlam{x}{ \hadd{9}{\hehole} }}
\end{align*}

If we have a function applied to a value, we step to the substitution as
usual:
\begin{align*}
  \pstep
      {
        \hap{\hlam{x}{\hadd{9}{\hehole}}}{5}
      }
      {
        \hadd{9}{\hehole}
      }
      \\
      \pstep
      {
        \hap{\hlam{x}{\hadd{x}{\hehole}}}{5}
      }
      {
        \hadd{5}{\hehole}
      }
\end{align*}

We also substitute inside non-empty holes: otherwise there would be a free
variable in the result, which would be nonsensical
\begin{align*}
  \pstep
      {
        \hap{\hlam{x}{\hadd{x}{\hhole{x}}}}{5}
      }
      {
        \hadd{5}{\hhole{5}}
      }
\end{align*}

The first non-trivial choice is to also evaluate inside the non-empty hole
when possible. The motivation for this is that a programmer may run a
partial program in order to gain insight into what they've written so
far. Evaluating terms terms that are not yet type-consistent with their
surroundings XXX.
\begin{align*}
  \pstep{
    \pstep
        {
          \hap{\hlam{x}{\hadd{x}{\hhole{\hadd{x}{x}}}}}{5}
        }
        {
          \hadd{5}{\hhole{\hadd{5}{5}}}
        }
  }{\hadd{5}{\hhole{10}}}
\end{align*}

\subsection{Possible Extensions}

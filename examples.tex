To motivate the purpose of a dynamics for partial programs, we consider a
series of examples of how terms with holes in them might step and why.

Since complete terms coincide exactly with the simply typed lambda
calculus, the evaluation semantics for them are totally
standard. Therefore, all the examples here are of terms with holes in
them. We explore the space of what could happen to partial terms during
evaluation by considering different structures.

These examples are written assuming an eager evaluation semantics. It's not
clear to us if that is important, or what the tradeoffs would be, so this
should be considered a stylistic detail that is not set in stone.

\subsection{Current Hazelnut}
First, we'll consider some examples that are within the small lambda
calculus that's given by Hazelnut.

Like in standard lambda calculus, function literals do not
evaluate further, reguardless of their contents:
\begin{align*}
  \pval{\hlam{x}{ \hadd{9}{\hehole} }}
\end{align*}

If we have a function applied to a value, we step to the substitution as
usual:
\begin{align*}
  \pstep
      {
        \hap{\hlam{x}{\hadd{9}{\hehole}}}{5}
      }
      {
        \hadd{9}{\hehole}
      }
      \\
      \pstep
      {
        \hap{\hlam{x}{\hadd{x}{\hehole}}}{5}
      }
      {
        \hadd{5}{\hehole}
      }
\end{align*}

We also substitute inside non-empty holes: otherwise there would be a free
variable in the result, which would be nonsensical
\begin{align*}
  \pstep
      {
        \hap{\hlam{x}{\hadd{x}{\hhole{x}}}}{5}
      }
      {
        \hadd{5}{\hhole{5}}
      }
\end{align*}

The first non-trivial choice is to also evaluate inside the non-empty hole
when possible. This follows from the motivation that a programmer may run a
partial program in order to gain insight into what they've written so far:
\begin{align*}
  \pstep{
    \pstep
        {
          \hap{\hlam{x}{\hadd{x}{\hhole{\hadd{x}{x}}}}}{5}
        }
        {
          \hadd{5}{\hhole{\hadd{5}{5}}}
        }
  }{\hadd{5}{\hhole{10}}}
\end{align*}

\subsection{Possible Extensions}
Now we consider a more extended example that's not within the language of
Hazelnut Instead, we use an ML-like syntax that should be understood
intuitively. We intend to extend the language to include constructs like
pattern matching, algebraic datatypes, and recursive functions. We believe
an understanding of the dynamics of such terms will help guide that
extension.

Imagine a programmer implementing $$\mathbf{map} : (\alpha \to \beta) \to
\alpha ~\mathit{list} \to \beta ~\mathit{list}$$ for the first time.
Provided with the informal specification that the function be applied to
every element of the list uniformly, she might arrive at the partial term
\begin{verbatim}
fun map (f : 'a -> 'b) (l : 'a list) : 'b list =
  case l
   of [] => <||>
    | x :: xs => (f x) :: <||>
\end{verbatim}
To see how this partial solution acts, she could then
\verb|map (fn x => x + 1) [1, 2]| to get

Alternatively, from the same task, the programmer might have produced the
partial program
\begin{verbatim}
fun map (f : 'a -> 'b) (l : 'a list) : 'b list =
  case l
   of [] => <||>
    | x :: xs => <||> :: (map f l)
\end{verbatim}
if they know the pattern of recursion they wish to follow, but aren't sure
what to do with each element yet. Here, evaluating the same term produces a
longer trace: because the body of the function is not yet final after
substitution, the function calls go through.

\begin{verbatim}
  trace goes here
\end{verbatim}

Finally, if our programmer had an idea about how to produce the answer for
each element but wasn't willing to commit to it yet, she might produce this
partial term:
\begin{verbatim}
fun map (f : 'a -> 'b) (l : 'a list) : 'b list =
  case l
   of [] => <||>
    | x :: xs => <|f x|> :: (map f l)
\end{verbatim}
This would produce a trace that reveals what the particular choice would
look like at each step without actually commiting to it---at which time
this would look like a very strong candidate implementation:
\begin{verbatim}
  trace goes here
\end{verbatim}

In Hazelnut, we describe a simple lambda calculus that allows holes to
appear in terms. The goal of this work is to give a formal description of
the process of editing programs---and to give static meaning to every edit
action on partially finished programs as well as every intermediate state
that arises in a sequence of edits.

To that end, holes in Hazelnut may appear in types as well as
expressions. The type system of Hazelnut gives static meaning to these
terms. On top of that we provide an editing semantics that describes
transformations between terms with holes in them, and prove a rich
metatheory about what edits are possible. Hazelnut's edit semantics uses a
notion of non-empty holes to allow programmers to write terms that are not
type consistent with their surroundings by delaying the type consistency
check until the programmer alerts the system that the term under
construction has been finished.

What Hazelnut does not provide is a way to \emph{run} programs before
they've been finished. In some sense, an evaluation semantics for terms
with holes in them is the ultimate debugging tool: if a programmer is
writing a program and she knows the general structure of what she wants to
write, but only has some idea of how to fill in all the holes, she can run
that program to see how each hole is actually used, and what information is
available at each step, to develop strong intuition about how to use that
information to achieve her goals.
